\section{Practical Considerations}
\label{sec:difficulties}

This section focuses on other things we had to do beyond
the core WPI algorithm presented in the previous section
in order to get a working tool. Its subsections should
be massaged to make them sound like theoretical problems---the idea
is that the reader should think of this section as the list of
problems we had to solve along the way, and the solutions to
the problems.

\subsection{Infinite Descending Chains}
\label{sec:infinite-descending-chains}

The usual lattice definition in an abstract interpretation
(or equivalently, a type system, cite Cousot 1997) forbids
infinite ascending chains but permits infinite descending chains.
The WPI fixpoint algorithm has a problem with them, though, and
needs widening operators. Talk about the early problems
with WPI on the Value Checker, where WPI would run for hundreds of
iterations: @IntRange(1, 10) -> @IntRange(1, 11) -> \ldots.

\subsection{Pre- and Post-conditions}
\label{sec:pre-post-conditions}

Discuss some of the troubles that Mike encountered when
he implemented pre- and post-condition support in WPI. Frame
this as a theoretical problem.

\subsection{Output Format}
\label{sec:output}

Find a way to frame the various WPI modes (i.e., JAIF mode,
stub mode, ajava mode) as a solution to a theoretical problem.

\todo{Think more about this and remember what other big problems
  we had to solve, and then add corresponding subsections.}
