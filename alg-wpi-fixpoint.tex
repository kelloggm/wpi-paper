% the core WPI fixpoint algorithm; that is, the outer WPI loop

\begin{algorithm}
  \DontPrintSemicolon
  \SetKwFunction{Infer}{infer}
  \SetKwProg{Fn}{def}{:}{}
  \SetKwInOut{Input}{input}
  \SetKwInOut{Output}{output}
  \Input{program $P$ and pluggable typechecker $T$}
  \Output{set of errors $E$ and set of type qualifiers $A$}
  \Fn{ \Infer{$P$, $T$} } {
%    \tcc{In each iteration, $\|prevA|$ is the specification set before running $T$, and $A$ is the set after.}
%    $\|prevA| \gets \emptyset, A \gets \emptyset$ \;
    $A \gets \emptyset$ \;
%    \tcc{\textsc{Instrument} modifies $T$ to both check the program and collect a candidate specification set.}
    $T_I \gets \textsc{EnableInference}(T)$ \;
%    $E, A \gets T_I(P, \|prevA|)$ \;
    \Repeat{$E = \emptyset \vee \|prevA| = A$}{
      % \tcc{Note that each element of $A$ is either a new type qualifier or refines some element of $\|prevA|$.}
      $\|prevA| \gets A$ \;
      $E, A \gets T_I(P, \|prevA|)$ \;
    }
    \Return $E, A$ \;
  }
  \caption{Iterated local type inference algorithm.  This is the ``outer
    loop'' of the approach, which iterates to a fixed point.
    The helper function \textsc{EnableInference} is defined by the modifications
    to the framework described in \cref{sec:instrument}.
    \todo{Why is the caption so narrow, not taking up the whole column?}
}
  \label{alg:wpi-fixpoint}
\end{algorithm}
