% the core WPI fixpoint algorithm; that is, the outer WPI loop

\begin{algorithm}[htbp]
  \DontPrintSemicolon
  \SetKwInOut{Input}{input}
  \SetKwInOut{Output}{output}
  \Input{program $P$ and pluggable typechecker $T$}
  \Output{Set of errors $E$ and set of specifications $A^{\prime}$}
  \BlankLine
%  \tcc{In each iteration, $A$ is the specification set before running $T$, and $A^{\prime}$ is the set after.}
  $A \gets \emptyset, A^{\prime} \gets \emptyset$ \;
%  \tcc{\textsc{Instrument} modifies $T$ to both check the program and collect a candidate specification set.}
  $T^{\prime} \gets \textsc{Instrument}(T)$ \;
  $E, A^{\prime} \gets T^{\prime}(P, A)$ \;
  \While{$E \neq \emptyset \wedge A \neq A^{\prime}$}{
    $A \gets A^{\prime}$ \;
    $E, A^{\prime} \gets T^{\prime}(P, A)$ \;
  }
  \Return $E, A^{\prime}$ \;

  \caption{The core fixpoint algorithm for our type inference system.
    The helper function \textsc{Instrument} is defined and explained in \cref{sec:instrument}.}
  \label{alg:wpi-fixpoint}
\end{algorithm}
