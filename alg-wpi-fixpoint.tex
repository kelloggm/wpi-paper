% the core WPI fixpoint algorithm; that is, the outer WPI loop

\begin{algorithm}
  \DontPrintSemicolon
  \SetKwFunction{Infer}{infer}
  \SetKwProg{Fn}{def}{:}{}
  \SetKwInOut{Input}{input}
  \SetKwInOut{Output}{output}
  \Input{program $P$ and pluggable typechecker $T$}
  \Output{set of errors $E$ and set of type qualifiers $A^{\prime}$}
  \Fn{ \Infer{$P$, $T$} } {
%    \tcc{In each iteration, $A$ is the specification set before running $T$, and $A^{\prime}$ is the set after.}
    $A \gets \emptyset, A^{\prime} \gets \emptyset$ \;
%    \tcc{\textsc{Instrument} modifies $T$ to both check the program and collect a candidate specification set.}
    $T^{\prime} \gets \textsc{EnableInference}(T)$ \;
    $E, A^{\prime} \gets T^{\prime}(P, A)$ \;
    \While{$E \neq \emptyset \wedge A \neq A^{\prime}$}{
      \tcc{Note that each element of $A^{\prime}$ is either a new type qualifier or refines some element of $A$.}
      $A \gets A^{\prime}$ \;
      $E, A^{\prime} \gets T^{\prime}(P, A)$ \;
    }
    \Return $E, A^{\prime}$ \;
  }
  \caption{The core fixpoint algorithm for our type inference system.
    The helper function \textsc{EnableInference} is defined by the modifications
    to the framework described in \cref{sec:instrument}.
    \todo{Why is the caption so narrow, not taking up the whole column?}
}
  \label{alg:wpi-fixpoint}
\end{algorithm}
