%%
%% This is file `sample-sigplan.tex',
%% generated with the docstrip utility.
%%
%% The original source files were:
%%
%% samples.dtx  (with options: `sigplan')
%% 
%% IMPORTANT NOTICE:
%% 
%% For the copyright see the source file.
%% 
%% Any modified versions of this file must be renamed
%% with new filenames distinct from sample-sigplan.tex.
%% 
%% For distribution of the original source see the terms
%% for copying and modification in the file samples.dtx.
%% 
%% This generated file may be distributed as long as the
%% original source files, as listed above, are part of the
%% same distribution. (The sources need not necessarily be
%% in the same archive or directory.)
%%
%% The first command in your LaTeX source must be the \documentclass command.
\documentclass[sigconf,screen,review,anon]{acmart}
\settopmatter{printccs=true, authorsperrow=4}

%%
%% \BibTeX command to typeset BibTeX logo in the docs
\AtBeginDocument{%
  \providecommand\BibTeX{{%
    \normalfont B\kern-0.5em{\scshape i\kern-0.25em b}\kern-0.8em\TeX}}}

%% Rights management information.  This information is sent to you
%% when you complete the rights form.  These commands have SAMPLE
%% values in them; it is your responsibility as an author to replace
%% the commands and values with those provided to you when you
%% complete the rights form.
% \setcopyright{acmcopyright}
% \copyrightyear{2018}
% \acmYear{2018}
% \acmDOI{10.1145/1122445.1122456}

%% These commands are for a PROCEEDINGS abstract or paper.
% \acmConference[Woodstock '18]{Woodstock '18: ACM Symposium on Neural
%   Gaze Detection}{June 03--05, 2018}{Woodstock, NY}
% \acmBooktitle{Woodstock '18: ACM Symposium on Neural Gaze Detection,
%   June 03--05, 2018, Woodstock, NY}
% \acmPrice{15.00}
% \acmISBN{978-1-4503-9999-9/18/06}


%%
%% Submission ID.
%% Use this when submitting an article to a sponsored event. You'll
%% receive a unique submission ID from the organizers
%% of the event, and this ID should be used as the parameter to this command.
%%\acmSubmissionID{123-A56-BU3}

%%
%% The majority of ACM publications use numbered citations and
%% references.  The command \citestyle{authoryear} switches to the
%% "author year" style.
%%
%% If you are preparing content for an event
%% sponsored by ACM SIGGRAPH, you must use the "author year" style of
%% citations and references.
%% Uncommenting
%% the next command will enable that style.
%%\citestyle{acmauthoryear}


% tables
\usepackage[utf8]{inputenc} 
\usepackage[T1]{fontenc}
\usepackage{microtype}
\usepackage{tabularx}
\usepackage{multirow}
\usepackage{booktabs}

\usepackage[ruled,vlined]{algorithm2e}

% margin notes
\usepackage{marginnote}

% xspace command
\usepackage{xspace}

% lstlisting command
\usepackage{listings}
\usepackage[scaled]{beramono}
\newcommand*\LSTfont{\Small\fontencoding{T1}\ttfamily\SetTracking{encoding=*}{-60}\lsstyle}
\lstset{language=Java,
  frame=none,
  aboveskip=1.5pt,
  belowskip=0pt,
  showstringspaces=false,
  columns=flexible,
  basicstyle=\LSTfont,
  numbers=none,
  numberstyle=\tiny\color{black},
  keywordstyle=\color{black},
  commentstyle=\color{black},
  stringstyle=\color{black},
  breaklines=true,
  breakatwhitespace=true,
  tabsize=3,
  %emph={@NonNegative,@Positive,@GTENegativeOne,@LTLengthOf,@LTEqLengthOf,@IndexFor,@IndexOrHigh,@IndexOrLow,@HasSubsequence,@LessThan,@SameLen,@SearchIndexFor,@MinLen,@ArrayLen,@IntVal,@IntRange,@LengthOf,@UpperBoundUnknown,@LowerBoundUnknown,int,double,List,Map,Object,SerialDate,Long,Integer,DefaultPolarItemRenderer,LegendItem,PolarPlot,XYDataset,long,T,String,string,byte,InputStream,CategoryDataset,DatasetRenderingOrder,ArrayList,Entry,Values,Number,ValuesContract,ImmutableIntArray,Dataset,XYZDataset}, emphstyle=\color{blue}
}

% Graphics
\usepackage{tikz}
\usetikzlibrary{arrows,automata,positioning}

% inference rules
\usepackage{mathpartir}

% Change font and line spacing for figure captions
% \usepackage{setspace,caption}
% \captionsetup{labelfont={small,bf}, textfont={small,bf,stretch=0.8}, labelsep=colon, margin=0pt}

\usepackage{balance} % balanced columns on last page

% cref command; best to load last
\usepackage{cleveref}
\newcommand{\crefrangeconjunction}{--}

%%% The following is specific to ESEC/FSE '21 and the paper
%%% 'Lightweight and Modular Resource Leak Verification'
%%% by Martin Kellogg, Narges Shadab, Manu Sridharan, and Michael D. Ernst.
%%%

\setcopyright{rightsretained}
\acmPrice{}
\acmDOI{10.1145/3468264.3468576}
\acmYear{2023}
\copyrightyear{2023}
\acmSubmissionID{foo}
\acmISBN{foo}
\acmConference[Under review '22]{Under review}{Date}{Location}
\acmBooktitle{Under review}

%%
%% end of the preamble, start of the body of the document source.
\begin{document}

\input{macros}

%%
%% The "title" command has an optional parameter,
%% allowing the author to define a "short title" to be used in page headers.
\title{Type Inference for Pluggable Typecheckers}
% or: \title{Inference of Pluggable Types}

%%
%% The "author" command and its associated commands are used to define
%% the authors and their affiliations.
%% Of note is the shared affiliation of the first two authors, and the
%% "authornote" and "authornotemark" commands
%% used to denote shared contribution to the research.

\author{Martin Kellogg}
\affiliation{New Jersey Institute of Technology, Newark
 \country{USA}}
\email{martin.kellogg@njit.edu}

\author{Michael D. Ernst}
\affiliation{University of Washington, Seattle
  \country{USA}}
\email{mernst@cs.washington.edu}

%%
%% By default, the full list of authors will be used in the page
%% headers. Often, this list is too long, and will overlap
%% other information printed in the page headers. This command allows
%% the author to define a more concise list
%% of authors' names for this purpose.
%\renewcommand{\shortauthors}{Trovato and Tobin, et al.}

%%
%% The abstract is a short summary of the work to be presented in the
%% article.
\begin{abstract}
  A pluggable type system extends a programming language with type qualifiers.
  It enables a programmer to write a type such as \<unsigned int>, \<secret
  string>, or \<nonnull object>.
  Type-checking with pluggable types detects and prevents more errors.
  The biggest obstacle to use of pluggable types is the need for
  programmers to write type qualifiers.
  Type inference can solve this problem.
  Traditional approaches to type inference are type-system-specific:
  build and then solve a system
  of constraints corresponding to the rules of the underlying type system.
  This approach does not scale well to pluggable type
  systems: for each new pluggable type system, the type inference algorithm
  must be extended to build and then solve constraints specific
  to that pluggable type system.

  We propose a novel type inference algorithm for pluggable type systems
  that is generic over the pluggable type system to which it is applied.
  That is, our system can infer type qualifiers for \emph{any} pluggable
  type system with no new\todo{This is a bit strong.  We do need custom
    code to deal with special cases (the framework has hooks for some of
    these), including non-qualifiers (declaration
    annotations, pre- and post-conditions.} type-system-specific code. The key insight behind
  our new technique is that extant, practical pluggable type systems are
  \emph{flow-sensitive}
  and therefore include \emph{local} type inference components. Our algorithm
  uses the results of local type inference to produce candidate summaries
  for procedures. These new summaries enable new local inference results
  elsewhere in the program---our algorithm runs until a fix-point in procedure
  summaries is reached.

  We have implemented our algorithm for the open-source Checker Framework
  project, which is widely used in industry and on which dozens of specialized pluggable typecheckers
  have been built. In experiments, with \todo{some number of} distinct
  pluggable typecheckers and \todo{some number of} projects, our algorithm
  inferred \todo{X}\% of the annotations that programmers wrote. We also compared our
  technique to other techniques for type inference.
\end{abstract}

%%
%% The code below is generated by the tool at http://dl.acm.org/ccs.cfm.
%% Please copy and paste the code instead of the example below.
%%
\begin{CCSXML}
<ccs2012>
<concept>
<concept_id>10011007.10011074.10011099</concept_id>
<concept_desc>Software and its engineering~Software verification and validation</concept_desc>
<concept_significance>500</concept_significance>
</concept>
</ccs2012>
\end{CCSXML}

%\ccsdesc[500]{Software and its engineering~Software verification and validation}

\ccsdesc[500]{Software and its engineering~Software verification}

%%
%% Keywords. The author(s) should pick words that accurately describe
%% the work being presented. Separate the keywords with commas.
\keywords{Pluggable type systems, type qualifiers, type checking, type inference, static analysis}

%% A "teaser" image appears between the author and affiliation
%% information and the body of the document, and typically spans the
%% page.
%% \begin{teaserfigure}
%%   \includegraphics[width=\textwidth]{sampleteaser}
%%   \caption{Seattle Mariners at Spring Training, 2010.}
%%   \Description{Enjoying the baseball game from the third-base
%%   seats. Ichiro Suzuki preparing to bat.}
%%   \label{fig:teaser}
%% \end{teaserfigure}

%%
%% This command processes the author and affiliation and title
%% information and builds the first part of the formatted document.

\maketitle

\section{Introduction}
\label{sec:intro}

\todo{Need to introduce the concepts of type qualifier, base/host type, and
  type (which consists of both); that types can be unqualified or written with a type
  annotation; that the checker does defaulting if the basetype is unannotated.}
\todo{Explain the distinction between a ``specification'' and a type qualifier.}

\todo{Should we be more careful to distinguish type annotations (a
  Java syntactic construct) from type qualifiers (a type-theoretic
  concept)?}

A pluggable type system~\cite{FosterFFA99} augments a host type system
with \emph{type qualifiers} that refine it.  A qualified type is
finer-grained than an unqualified one and therefore gives more precise
information about what values are possible at run time.
Researchers have devised practical pluggable type systems
to prevent numerous kinds of defects, including null-pointer
dereferences~\cite{BanerjeeCS2019,PapiACPE2008,DietlDEMS2011},
array-bounds violations~\cite{KelloggDME2018},
violations of locking discipline~\cite{ErnstLMST2016},
mutations of immutable data~\cite{DietlDEMS2011,PapiACPE2008,coblenz2017glacier},
and others.
A successful type-checking run proves that the undesirable behavior will
never occur.
Pluggable type systems are a standard practice in industry; for example,
they are used at Amazon~\cite{KelloggSTE2020,KelloggRSSE2020},
Google~\cite{SadowskiAEMCJ2018}, and Uber~\cite{BanerjeeCS2019}

Pluggable types are an attractive verification and bug-finding strategy
because programmers
are familiar with type systems and are used to writing types.
% because
% they are programming in a host language like Java or C\# with static
% types.
The type qualifiers also serve as concise, machine-checked documentation.
However, writing type qualifiers in a legacy codebase
is time-consuming and intimidating for developers.

\todo{Start out talking about languages like ML and Haskell, which are
  strongly typed but use whole-program type inference.  It's towers of
  exponential time in the (very rare) worst case.}
This problem also applies to non-pluggable type systems: some developers
prefer untyped or dynamically-typed languages like Python or JavaScript
because they do not want to write type annotations.
%
The traditional approach to solving this problem is \emph{type inference}:
deducing the proper types for a program from the program's structure
by solving a set of constraints induced by type uses.
%
Type inference systems for languages with Hindley-Milner type systems
are well-developed~\cite{DamasM1982} (\todo{add more recent citations, maybe
  say something about more complex dependent type systems?}).
%
Type inference for popular dynamic languages like Python is an open
research problem. Researchers have proposed approaches based
on MaxSAT~\cite{hassan2018maxsmt} and machine learning~\cite{xu2016python,peng2022static}.

Unfortunately, none of these approaches is practical for pluggable types.
Building a type inference algorithm in the style of Algorithm W for each
type system is impractical: there are too many pluggable type
systems.\todo{I don't buy this.  There aren't too many type systems to
  implement.  Doing a bit of extra work to define inference for each is a
  constant factor overhead over writing the type checker for each.}
A similar problem occurs when trying to build a constraint system to
dispatch to an SMT solver: the type rules of the specific pluggable type system
need to be encoded in the constraint-generation rules. Finally, approaches
based on machine learning lack sufficient training data for
many pluggable type systems, they give no guarantee of correctness, and may
not be able to explain their choices to the programmer.
%
We desire an
inference algorithm that is \emph{generic} over the pluggable type system
to which it is applied: the type system designer should not need to
modify their type system implmentation in order to access
the benefits of type inference.

To that end, we propose a general type inference algorithm for pluggable type
systems that is applicable to any flow-sensitive pluggable type system.
Our key insight is that
practical frameworks for building pluggable type systems already provide
local type inference in the form of flow-sensitivity within the body
of methods.
We modify the framework to record inferred method/class/field summaries.
Our approach iteratively type-checks the program, using and improving the
summaries, until reading a fixed point.
That fixed point is a candidate set of type qualifiers, which are
consistent with the program.

If the program typechecks with the candidate set of qualifiers, the program
is correct with respect to that type system.
If not, then the program contains a defect, or it is not internally
consistent, or its correctness is beyond the capabilities of the
type-checker (this is when a programmer would write a cast).
In any event, the inferred type qualifiers can help the programmer.
In our implementation, the user (a programmer) can decide whether to insert
the type qualifiers in the source code, or to store them in a side file.



We implemented this idea for the Checker Framework~\cite{PapiACPE2008},
a popular open-source pluggable type system
framework for Java.
%
\todo{There were some difficulties along the way---we're going to
impress you with our clever solution to tricky problems, and also how much
engineering we did.}
%
We used the extended Checker Framework to run \todo{X} different pluggable typecheckers
on \todo{some massive number of} lines of code to demonstrate that
our approach is both effective and general.
In experiments, our inference approach inferred \todo{Y\%} of human-written
type qualifiers.

\todo{Decide if we're actually going to do this.}
We also compared our approach to bespoke inference systems that build
systems of constraints for particular type systems, like nullness or
maybe CFI (if Werner is an author of this paper). We show that our approach
produces similarly-good results, but didn't require a ton of extra
implementation effort for each new type system.

Our contributions are:
\begin{itemize}
\item a novel type inference algorithm for flow-sensitive pluggable
  typecheckers (\cref{sec:core-algorithm});
\item a collection of enhancements to the algorithm that are necessary to
  make it practical (\cref{sec:difficulties});
\item an implementation of our new type inference algorithm within a framework
  for building pluggable typecheckers (\cref{sec:implementation});
\item an evaluation of our implementation, that shows that it can infer
  \todo{X}\% of human-written annotations in \todo{Y} projects totalling
  \todo{Z} lines of non-comment, non-blank Java code, across \todo{W} different
  pluggable typecheckers (\cref{sec:evaluation}); and,
\item a comparison of our generic algorithm to specialized inference
  techniques for specific typecheckers, which demonstrates that our generic
  approach is about as good at inferring annotations but requires less
  custom code (\cref{sec:comparison}).
\end{itemize}


\section{Background and Motivating Example}
\label{sec:motivating-example}

A type is a set of run-time values.
% A static type overestimate possible run-time values. 
A \textit{type qualifier}~\cite{FosterFFA99}
is a restriction on a type that limits which run-time values
the qualified type can represent. For example, \<positive int>
is a qualified type: \<positive> is the type qualifier, and \<int>
is the base type.
%
A pluggable type system consists of a hierarchy of type qualifiers.

Practical pluggable type systems are flow-sensitive.
For example, after an assignment \codeid{x.f = somePositive} or a test
\codeid{x.f > 0}, the type of \<x.f>
might change from \<int> to \<positive int>
until a possible side effect or a control flow join.
%
Each pluggable typechecker is therefore effectively
an abstract interpretation~\cite{Cousot1997}, with the
abstract interpretation's lattice being equivalent to
the type qualifier hierarchy.

Like the host type systems to which they are applied,
pluggable type systems are modular:  they can be run incrementally on a
procedure or a file at a time, rather than requiring a whole-program
analysis.

For example, consider a pluggable type system
designed to prevent negative array accesses.
It would require that the type of any array access is non-negative.
The following code does not type-check as written:

\begin{Verbatim}
// Returns the value in a at the index. For this procedure,
// the first element of the array is at index 1.
string getOneIndexed(string[] a, int index) {
  return a[i - 1];
}
\end{Verbatim}

Because \<index>'s type \<int> is an unqualified type from the host language
(\ie a ``base type''),
a pluggable typechecker would \emph{default}
it, most likely to a worst-case assumption that \<index> could
be any integer (the ``top qualifier'' or $\top$).
%
With this type, every call to \<getOneIndexed> would typecheck, but its
body would not.


To make the code typecheck (equivalently, to verify that its array accesses
are not at negative indices)
a programmer would write the formal parameter as
\<positive int index>.

Within the procedure body, the typechecker \emph{relies} on the fact that
\<index> has type <positive int>.  At call sites, the typechecker
\emph{guarantees} that only postive integers are passed as arguments.  So
long as every procedure is checked, this approach is sound.

The typechecker is modular:  it can visit each procedure
once\footnote{While type-checking is modular, type inference is a
super-linear whole-program analysis.}
and never has to reason across procedure
boundaries, but only to use the specification (the types) of called procedures.

The annotation burden scales linearly with the size of the code
base, which may be large for legacy code.  Furthermore,
many pluggable type systems require significant numbers of annotations.
For example, a pluggable type system for preventing out-of-bounds array accesses
required
one type qualifier for every 32 lines of non-comment, non-blank code~\cite{KelloggDME2018}.

Our goal in this work is to avoid the burden of writing these type qualifiers
by automatically taking a flow-sensitive, modular pluggable typechecker
like the ones that exist in practice and transforming it into one that performs
inter-procedural inference.

\todo{Discuss the difference between type inference and type
  reconstruction?  Probably just give a nod to type reconstruction in the
  related work.  Our experiments are type reconstruction.}


\section{Type Inference Algorithm}
\label{sec:algorithm}

This section presents our type inference algorithm. This algorithm is
independent of the underlying pluggable typechecker: that is,
it applies equally well to any pluggable typechecker that performs flow-sensitive
local type inference.

\todo{Should we call $T_I$ an ``inferring type-checker''?  Having a name can
be handy.}

\todo{It would be nice to give a name to our approach, rather than always
  calling it ``our approach''.  How about ``iterated local type
  inference''?  Do we have any better ideas?}

Our key observation is that practical pluggible type-checkers use 
flow-sensitive type refinement, which is a type of local type inference.
\todo{We must define the above terms.}
The two key ideas behind our type inference approach are to expose the
results of the local type inference, and to iteratively call a type-checker
reusing these results.

Instead of building a type inference for each type system,
our key idea is to modify the underlying \emph{framework}
on which the pluggable typecheckers are built.  This converts any
type-checker built on the framework into an \emph{inferring type-checker}.

% Pluggable
% typecheckers use a modified version of the host type system
% that supports type qualifiers. Our approach modifies this support
% layer for pluggable typechecking in order to support inference for
% any typechecker.

A typechecker $T : P \rightarrow E$
takes a program and outputs a (possibly empty) set of
type errors.  An inferring typechecker using our modified framework,
$T_I : \langle P, A \rangle \rightarrow \langle E, A' \rangle$,
takes a program along with a set of additional type qualifiers $A$.
It outputs errors and
inferences (\ie $A'$, a new set of type qualifiers).
The errors $E$ are exactly those $T$ would output, if the
type qualifiers in $A$ had been written on $P$ by a programmer.
$P$ may or may not contain programmer-written type qualifiers.

\todo{Relate to the two ``key ideas'' above.}
\todo{Why in this order?  It's opposite what is described above.}
We describe the algorithm in two parts: first,
\Cref{sec:core-algorithm} gives the fixpoint
algorithm used to infer types for a particular program
using our modified typechecking framework.
\Cref{sec:instrument} explains the modifications to the pluggable
typechecking framework that enable type inference for a pluggable
typechecker $T$ (\ie convert it to $T_I$ as described above).

Note that in these two phases of our algorithm, estimates for
candidate type qualifiers move in different directions in the
type qualifier hierarchy.
%
In the outer loop (\cref{sec:core-algorithm}),
estimates only go down (or stay that same): that is,
type qualifiers become more refined.
%
In a single round of type inference, however, type qualifier
estimates go up in the hierarchy: for example, the estimate
for the type of a field is the least upper bound of the types
of all the expressions that are assigned into that field.

%% \todo{Here is a fact that I think we should explain to readers.
%%   Within a single type-checking run, the type estimates only go up.
%%   Between runs (that is, in \cref{alg:wpi-fixpoint}), the type estimates
%%   only go down.}


\subsection{Fixpoint Algorithm:  the outer loop}
\label{sec:core-algorithm}

% the core WPI fixpoint algorithm; that is, the outer WPI loop

\begin{algorithm}
  \DontPrintSemicolon
  \SetKwFunction{Infer}{infer}
  \SetKwProg{Fn}{def}{:}{}
  \SetKwInOut{Input}{input}
  \SetKwInOut{Output}{output}
  \Input{program $P$ and pluggable typechecker $T$}
  \Output{set of errors $E$ and set of type qualifiers $A^{\prime}$}
  \Fn{ \Infer{$P$, $T$} } {
%    \tcc{In each iteration, $A$ is the specification set before running $T$, and $A^{\prime}$ is the set after.}
    $A \gets \emptyset, A^{\prime} \gets \emptyset$ \;
%    \tcc{\textsc{Instrument} modifies $T$ to both check the program and collect a candidate specification set.}
    $T^{\prime} \gets \textsc{EnableInference}(T)$ \;
    $E, A^{\prime} \gets T^{\prime}(P, A)$ \;
    \While{$E \neq \emptyset \wedge A \neq A^{\prime}$}{
      \tcc{Note that each element of $A^{\prime}$ is either a new type qualifier or refines some element of $A$.}
      $A \gets A^{\prime}$ \;
      $E, A^{\prime} \gets T^{\prime}(P, A)$ \;
    }
    \Return $E, A^{\prime}$ \;
  }
  \caption{The core fixpoint algorithm for our type inference system.
    The helper function \textsc{EnableInference} is defined by the modifications
    to the framework described in \cref{sec:instrument}.
    \todo{Why is the caption so narrow, not taking up the whole column?}
}
  \label{alg:wpi-fixpoint}
\end{algorithm}


The inference algorithm appears
in \cref{alg:wpi-fixpoint}.   It iteratively analyzes
the target program $P$ with an inferring version of the
pluggable typechecker
% whose core qualified type rules
% have been modified to support inference in the manner described in
(see \cref{sec:instrument}),
% recording intermediate results at each
% step (the sets of inferred type qualifiers $A$ and $A^{\prime}$)
until
either there are no remaining typechecking errors
(\ie $E = \emptyset$)
or the
type qualifiers reach fixpoint (\ie $\|prevA| = A$).

%% This fixpoint loop is quite general: its success for our
%% purposes depends heavily on the \textsc{Instrument} procedure
%% that outputs locally-inferred specifications. \Cref{sec:instrument}
%% describes how we implement \textsc{Instrument} in a way that is
%% applicable to any pluggable typechecker.

\begin{theorem}
\Cref{alg:wpi-fixpoint} monotonically refines types.  That is,
each element $a$ of $A$ is either 1) not present in $\|prevA|$, or
2) there exists some element $\|preva|$ in $\|prevA|$ such that $a \sqsubseteq \|preva|$
and both $\|preva|$ and $a$ qualify the same base type in the same program location.
\end{theorem}

\begin{proof}
By induction on the type rules in \cref{sec:instrument}.
\end{proof}


\subsubsection{Soundness}
\label{sec:soundness}

Any infer-then-check approach is sound so long as the ``check'' step is sound.
Even if the inference algorithm were to produce incorrect type qualifiers,
the checker would reject them.


\subsection{Modifications to the Typechecking Framework}
\label{sec:instrument}

The algorithm presented in \cref{sec:core-algorithm} works for
any pluggable typechecker that supports flow-sensitive inference:
that is, it does not require a type system implementer to write
any special rules to support type inference. Instead, this section
describes our approach to \emph{automatically} modify a given
pluggable typechecker to support inference, corresponding to the
\textsc{Enable\-Inference} helper function in \cref{alg:wpi-fixpoint}.

The key idea behind our inference approach to instrumenting the typechecker
is to modify the \emph{framework}.  Once the pluggable type-checking
framework is modified, inference is enabled for every typechecker built on it.
Our modifications
can be conceptualized at the type-qualifier-theory level: that is,
we modify the rules for typechecking \emph{any} pluggable type system
so that inference is supported,
regardless of the particular qualifiers it happens to support.

Our modifications rely on the fact that all practical pluggable type systems
do \emph{local}, intra-procedural flow-sensitive type inference.
This means that programmers rarely need to write annotations within method bodies.
Analogously, Java's \<var> keyword permits programmers to omit Java basetypes within
method bodies.
Programmers are more willing to write types on method
signatures, where they form valuable documentation.  (Our goal is to lift
even this burden, for type annotations)

\todo{Also mention
  that this seems to be a general trend in language design, citing maybe Kotlin?}

A pluggable type \emph{checker} permits programmers to leave basetypes unqualified.
On APIs (class/method/field declarations), the type checker uses defaulting
rules to assign a qualifier to each unqualified basetype.
Within a code block, the type checker performs local flow-sensitive type
inference.


\begin{figure*}
  \begin{mathpar}
    \inferrule* [right=INVOKE]
                {
                  \std{\Gamma \vdash m(f_0,\ldots,f_n) : }~\qual{q_R}~\std{\tau_R}
                  \\
                  \std{\Gamma \vdash \forall i \in 0,\ldots,n . ~f_i :~} \qual{q_{F_i}}~\std{\tau_{F_i}}
                  \\
                  \std{\Gamma \vdash \forall i \in 0,\ldots,n . ~e_i :~} \qual{q_{A_i}}~\std{\tau_{A_i}}
                  \\
                  \std{\Gamma \vdash \forall i \in 0,\ldots,n . ~} \qual{q_{A_i}}~\std{\tau_{A_i}~\sqsubseteq}~\qual{q_{F_i}}~\std{\tau_{F_i}}
                  \\
                  \infr{\infEnv \vdash \forall i \in 0,\ldots,n . ~f_i~:~q_{I_i}~\tau_{F_i}}
                }
                {
                  \std{\Gamma \vdash m(e_0,\ldots,e_n) : }~\qual{q_R}~\std{\tau_R}
                  \\
                  \infr{\infEnv \vdash \forall i \in 0,\ldots,n . ~f_i~:~\mathit{LUB_Q}(q_{A_i},~q_{I_i})~\tau_{F_i} }
                }
                
     \inferrule* [right=NEW]
                {
                  \std{\Gamma \vdash \<new T>(f_1,\ldots,f_n) : }~\qual{q_R}~\std{\tau_R}
                  \\
                  \std{\Gamma \vdash \forall i \in 1,\ldots,n . ~f_i :~} \qual{q_{F_i}}~\std{\tau_{F_i}}
                  \\
                  \std{\Gamma \vdash \forall i \in 1,\ldots,n . ~e_i :~} \qual{q_{A_i}}~\std{\tau_{A_i}}
                  \\
                  \std{\Gamma \vdash \forall i \in 1,\ldots,n . ~} \qual{q_{A_i}}~\std{\tau_{A_i}~\sqsubseteq}~\qual{q_{F_i}}~\std{\tau_{F_i}}
                  \\
                  \infr{\infEnv \vdash \forall i \in 1,\ldots,n . ~f_i~:~q_{I_i}~\tau_{F_i}}
                }
                {
                  \std{\Gamma \vdash \<new T>(e_1,\ldots,e_n) : }~\qual{q_R}~\std{\tau_R}
                  \\
                  \infr{\infEnv \vdash \forall i \in 1,\ldots,n . ~f_i~:~\mathit{LUB_Q}(q_{A_i},~q_{I_i})~\tau_{F_i} }
                }

     \inferrule* [right=FORMAL-ASSIGN]
                {
%                  \std{f~is~a~formal~parameter} \\
                  \std{\Gamma \vdash f~:}~\qual{q_F}~\std{\tau_F} \\
                  \std{\Gamma \vdash e~:}~\qual{q_A}~\std{\tau_A} \\
                  \std{\Gamma \vdash~} \qual{q_A}~\std{\tau_A~\sqsubseteq}~\qual{q_F}~\std{\tau_F} \\
                  \infr{\infEnv \vdash f~:~q_I~\tau_F}
                }
                {
                  \std{\Gamma \vdash f~:=~e} \\
                  \infr{\infEnv \vdash f~:~\mathit{LUB_Q}(q_A, q_I)~\tau_F}
                }

     \inferrule* [right=FIELD-ASSIGN]
                {
%                  \std{f~is~a~formal~parameter} \\
                  \std{\Gamma \vdash x.f~:}~\qual{q_F}~\std{\tau_F} \\
                  \std{\Gamma \vdash e~:}~\qual{q_A}~\std{\tau_A} \\
                  \std{\Gamma \vdash~} \qual{q_A}~\std{\tau_A~\sqsubseteq}~\qual{q_F}~\std{\tau_F} \\
                  \infr{\infEnv \vdash C.f~:~q_I~\tau_F}
                }
                {
                  \std{\Gamma \vdash x.f~:=~e} \\
                  \infr{\infEnv \vdash C.f~:~\mathit{LUB_Q}(q_A, q_I)~\tau_F}
                }

     \inferrule* [right=RETURN]
                {
%                  \std{f~is~a~formal~parameter} \\
                  \std{\Gamma \vdash m(f_0,\ldots,f_n)~:}~\qual{q_R}~\std{\tau_R} \\
                  \std{\Gamma \vdash e~:}~\qual{q_A}~\std{\tau_A} \\
                  \std{\Gamma \vdash~} \qual{q_A}~\std{\tau_A~\sqsubseteq}~\qual{q_R}~\std{\tau_R} \\
                  \infr{\infEnv \vdash m(f_0,\ldots,f_n)~:~q_I~\tau_R}
                }
                {
%                  \std{\Gamma \vdash \<return>~e} \\
                  \std{\<return>~e \in m} \\
                  \infr{\infEnv \vdash m(f_0,\ldots,f_n)~:~\mathit{LUB_Q}(q_A, q_I)~\tau_R}
                }

    \inferrule* [right=OVERRIDE]
                {
                  % return types
                  \std{\Gamma \vdash m_B(f_{0_B},\ldots,f_{n_B}) : }~\qual{q_{R_B}}~\std{\tau_{R_B}}
                  \\
                  \std{\Gamma \vdash m_P(f_{0_P},\ldots,f_{n_P}) : }~\qual{q_{R_P}}~\std{\tau_{R_P}}
                  \\
                  \std{\Gamma \vdash~} \qual{q_{R_B}}~\std{\tau_{R_B}~\sqsubseteq}~\qual{q_{R_P}}~\std{\tau_{R_P}}
                  \\
                  \std{\Gamma \vdash \forall i \in 0,\ldots,n_B . ~f_{B_i} :~} \qual{q_{B_i}}~\std{\tau_{B_i}}
                  \\
                  \std{\Gamma \vdash \forall i \in 0,\ldots,n_P . ~f_{P_i} :~} \qual{q_{P_i}}~\std{\tau_{P_i}}
                  \\
                  \std{\Gamma \vdash \forall i \in 0,\ldots,n_B . ~} \qual{q_{B_i}}~\std{\tau_{B_i}~\sqsubseteq}~\qual{q_{P_i}}~\std{\tau_{P_i}}
                  \\
                  \std{\vdash n_B~=~n_P}
                  \\
                  \infr{\infEnv \vdash m_1(f_{0_B},\ldots,f_{n_B})~:~q_{R_B-I}~\tau_{R_B}}
                  \\
                  \infr{\infEnv \vdash m_P(f_{0_P},\ldots,f_{n_P})~:~q_{R_P-I}~\tau_{R_P}}
                  \\
                  \infr{\infEnv \vdash \forall i \in 0,\ldots,n_B . ~f_{B_i}~:~q_{B_i-I}~\tau_{B_i}}
                  \\
                  \infr{\infEnv \vdash \forall i \in 0,\ldots,n_P .~f_{P_i}~:~q_{P_i-I}~\tau_{P_i}}
                }
                {
                  \std{\Gamma \vdash m_B(f_{0_B},\ldots,f_{n_B})~\mathit{is~a~valid~override~of}~m_P(f_{0_P},\ldots,f_{n_P})}
                  \\
                  \infr{\infEnv \vdash m_P(f_{0_P},\ldots,f_{n_P})~:~\mathit{LUB_Q}(q_{R_B-I}, q_{R_P-I})~\tau_{R_P}}
                  \\
                  \infr{\infEnv \vdash \forall i \in 0,\ldots,n_P . ~f_{P_i}~:~\mathit{LUB_Q}(q_{B_i-I},~q_{P_i-I})~\tau_{P_i} }
                }
                
  \end{mathpar}

  \todo{These rules do not seem to handle qualifier polymorphism, where the
    return type can depend on the instantiation.  Somewhere the paper
    should discuss this, and whether our inference algorithm can handle it
    (maybe only when written by the programmer?)
    and which types of polymorphism our algorithm can handle and the
    challenges thereto.  This relevant to both \textsc{INVOKE} and \textsc{RETURN}.}

  \todo{If you want to save space, you could drop \textsc{NEW} and say it
    is analogous to \textsc{INVOKE}.  Leaving it in makes the figure
    bigger, which could be a plus or a minus depending on your perspective.}

  \todo{Somewhere, we should show some type-system-specific rules like
    those for \<x.f> and \<a[i]>.}

  \todo{There is a typographic convention that you can use to eliminate all
    the $\forall$ from this diagram.  
$\overline{f_i : q_{F_i}~\tau_{F_i}}$ is shorthand for
$\forall i . ~f_i :~ q_{F_i}~\tau_{F_i}$.  I think this
    would make the diagram easier to read.}

  \caption{Modified type rules used by iterated local type inference.
    Applying these type rules once to every statement in a program
    constitutes the ``inner loop'' of the inference algorithm.
    \std{Gray} indicates
    standard type rules for an object-oriented language with Java-like syntax. \qual{Black} indicates additions to support
    pluggable typechecking\todo{cite (Foster, I guess)}. \infr{Red} indicates additions to support inference, \ie our
    contribution in this paper.
    Throughout, ``R'' subscripts refer to return types; ``F'' to formal parameters; ``A'' to
    actual arguments; and ``I'' to inference results.
    \todo{Which of the qualifier variables can be ``unqualified''?  I guess
      it is all of them?}
    In an assignment \<x=y>, \<x> is the ``formal'' and \<y> is the ``actual''.
    In the \textsc{OVERRIDE} rule, the subscripts ``B'' and ``P'' are mneumonics
    for ``suBtype'' and ``suPertype'', referring the overriding method and the overidden
    method, respectively.
    Type rules that do not require modification to support inference
    are elided for space.}
  \label{fig:type-rules}
\end{figure*}

The modified type rules appear in \cref{fig:type-rules}.
%% This is in the caption, so Mike doesn't think it needs to be in the body text.
% \std{Gray} indicates
% standard type rules for an object-oriented language with Java-like
% language.  \qual{Black} indicates additions to support
% pluggable typechecking. \infr{Red} indicates additions to support inference, \ie our
% contribution in this paper.

$m(f_0,~\ldots,~f_n)$ refers to a method declaration: $m$ is a method name,
and each $f_i$ is a formal parameter declaration.
\todo{Here and in \cref{fig:type-rules}, $f_i$ is ambiguous.  It appears in
  $m(\ldots, f_i, \ldots)$, but it is also mentioned as $f_i~:~q_F~\tau_F$.
  So, does $f_i$ stand for just the name (implied by the latter), or for
  the name and the qualified type (implied by the former)?  Please clarify,
  and use different variables for the two concepts.}
The syntax $m(f_0,~\ldots,~f_n)~:~q_R~\tau_R$ means that $m$'s (qualified) return
type is $q_R~\tau_R$.  The syntax $f_i~:~q_F~\tau_F$ means that the declared
type of the formal parameter $f_i$ (of $m$) is $q_F~\tau_F$.
%
Type rules that do not need to be modified to support inference are elided for space.

$\infEnv$ is the \emph{inference environment}, similar to the standard (qualified)
typing environment $\Gamma$. $\Gamma$ maps expressions and declarations to qualified types.
$\infEnv$ maps declarations to possibly-qualified types.
$\infEnv$ only maps declarations because our inference procedure does not need to infer
types for expressions: we assume that $\Gamma$ already does so via flow-sensitive
refinement. Rather, $\infEnv$'s purpose is to map declarations to the results of inference.
Unlike $\Gamma$, the values in $\infEnv$ are \emph{possibly-qualified}, meaning that they can either
be qualified types or unqualified types. Initially, $\infEnv$ contains qualified types only
for declarations that were qualified before the current type-checking pass of inference (which may come from
the programmer or from a previous type-checking round in the \cref{alg:wpi-fixpoint} fixpoint loop).
Once every statement in the program has been type-checked, the current
round of inference terminates.  Its result is all mappings to qualified
types in $\infEnv$.  Any type that remains unqualified
throughout inference is not annotated, because no information about it was learned.

The function $\mathit{LUB_Q}(q_1, q_2)$ accounts for possibly-qualified types.
$q_1$ and $q_2$ are each either a type qualifier or ``not present''.
If both arguments are qualifiers, then the result of $\mathit{LUB_Q}$
is their least upper bound. If only one qualifier is present, then $\mathit{LUB_Q}$'s result
is that qualifier; if both qualifiers are not present, $\mathit{LUB_Q}$'s result is ``not present'',
resulting in an unqualified result.

\todo{Then, this section should describe the interesting parts of the
  modified type rules in detail? At a minimum, we should probably walk the reader through
  one of the type rules.}

\todo{Within a type-checking round (the ``inner loop''), estimates always
  go up.}


\subsubsection{Completeness}
\label{sec:complete}

Note that this inference system is intentionally \emph{not} complete: it
does not and cannot infer all possibly-true type qualifiers for a given type
system. To see why not, note that type qualifiers on e.g., formal parameters
are inferred from the actual types of the arguments at call sites. If there
are no call sites for a method in a given program, then the \textsc{INVOKE}
rule will never be fired: no information will be inferred for those formal
parameters\todo{not by \textsc{INVOKE}, but \textsc{FORMAL-ASSIGN} might},
and they will remain unqualified.

The lack of completeness is by design. Recall that our goal is not a
set of type qualifiers that perfectly captures whatever facts are true
about the program, but rather a set of type qualifiers that is useful
\emph{in practice} for typechecking programs. Not all true type qualifiers
are useful; as our experiments show, this system still produces many more
(true) type qualifiers than a human would write. \todo{Add a forward ref
  to the prior sentence.} Another benefit of completeness being a non-goal
is that our inference system is permitted to not infer a type qualifier
in practically any scenario where doing so might lead to sub-optimal results;
for example, see our handling of recursion (\cref{sec:infinite-descending-chains}).

% LocalWords:  typechecker typechecked typechecking typecheckers intra
% LocalWords:  decl standard''


\section{Practical Considerations}
\label{sec:difficulties}

This section focuses on other things we had to do beyond
the core WPI algorithm presented in the previous section
in order to get a working tool. Its subsections should
be massaged to make them sound like theoretical problems---the idea
is that the reader should think of this section as the list of
problems we had to solve along the way, and the solutions to
the problems.

\subsection{Infinite Descending Chains}
\label{sec:infinite-descending-chains}

The usual lattice definition in an abstract interpretation
(or equivalently, a type system, cite Cousot 1997) forbids
infinite ascending chains but permits infinite descending chains.
The WPI fixpoint algorithm has a problem with them, though, and
needs widening operators. Talk about the early problems
with WPI on the Value Checker, where WPI would run for hundreds of
iterations: @IntRange(1, 10) -> @IntRange(1, 11) -> \ldots.

\subsection{Pre- and Post-conditions}
\label{sec:pre-post-conditions}

Discuss some of the troubles that Mike encountered when
he implemented pre- and post-condition support in WPI. Frame
this as a theoretical problem.

\subsection{Output Format}
\label{sec:output}

Find a way to frame the various WPI modes (i.e., JAIF mode,
stub mode, ajava mode) as a solution to a theoretical problem.

\todo{Think more about this and remember what other big problems
  we had to solve, and then add corresponding subsections.}


\section{Implementation}
\label{sec:implementation}

We built it and it works. Say things about the Checker Framework. This section
should be pretty short; it might give some more details about how we solved
some of the theoretical problems we describe in \cref{sec:difficulties}?


\section{Evaluation}
\label{sec:evaluation}

This section is our main evaluation. Our evaluation has two goals:
\begin{itemize}
\item show that WPI is \emph{effective}: it can infer most of the
  annotations that humans have written in prior case studies.
\item show that WPI is \emph{general}: without any extra per-checker work,
  we can apply it to lots of checkers.
\end{itemize}

%% This is the primary table.

\begin{table*}
  \caption{
    WPI performance on human-annotated benchmarks. ``Annotated for'' is
    the names of the typecheckers for which human annotations were written
    (see \cref{tab:checkers} to understand the codes).
    ``NCNB LoC'' is the number of lines of non-comment, non-blank code.
    ``Annos.'' is the number of human-written annotations, across all typecheckers.
    ``WPI \% Inferred'' is the percentage of the human-written annotations that WPI
    inferred.\todo{What if WPI inferred something weaker or stronger?  How
      does that count?}\todo{Does it make sense to measure precision as
      well as recall?}\todo{Does it make sense to record the number of
      type-checking errors?  I don't think so:  the number of wrong
      inferred annotations is a better measure of remaining human effort.}
  }
  \label{tab:case-studies}
  \posttablecaption

  \begin{tabular}{@{}ll|rrr@{}}
    \textbf{\smaller{Benchmark}} & Annoted for & NCNB LoC & Annos. & WPI \% Inferred \\
    \textbf{\smaller{JFreeChart}} & In & 95,000 & 2,500 & X\% \\
    \textbf{\smaller{PlumeUtil}} & All & 10,000 & ? & X\% \\
    \ldots & \ldots & \ldots & \ldots & \ldots \\
  \end{tabular}
\end{table*}


Our methodology is to collect a set of projects that have been annotated
by humans, remove the annotations, and then run WPI.
The results appear in \cref{tab:case-studies}.

%% This is the table per-checker.

\begin{table*}
  \caption{
    The results of \cref{tab:case-studies} indexed by typechecker
    rather than by benchmark. ``Code'' is the two-character code
    used in the ``Annotated for'' column of \cref{tab:case-studies}.
    A code in \cref{tab:case-studies} of ``All'' means that every
    checker in this table was run on that project.\todo{Do we need the
      ``All'' code?}
  }
  \label{tab:checkers}
  \posttablecaption

  \begin{tabular}{@{}ll|rrrr@{}}
    \textbf{\smaller{Checker}} & Code & \# projects & NCNB LoC & Annos. & WPI \% Inferred \\
    \textbf{\smaller{Index}} & In & 2 & 120,000 & 3,200 & X\% \\
    \textbf{\smaller{Nullness}} & Nu & 5 & 100,000 & ? & X\% \\
    \ldots & \ldots & \ldots & \ldots & \ldots \\
  \end{tabular}
\end{table*}


We also broke out the results by checker, in \cref{tab:checkers}. We discuss
the results.


\section{Comparison to Other Tools}
\label{sec:comparison}

In this section, we compare our generic approach to
checker-specific inference tools. We probably compare
on our nullness benchmarks against the nullness-specific
tools. We might compare against CFI.

We hope that the results show that WPI is about as good as
the other approaches. The relative advantage of WPI is that
it doesn't require any custom code, unlike the others.

A table presents the results.


Here are potential tools:
\begin{itemize}
\item
  Inference of \<@Nullable>: the AnnotateNullable tool of the Daikon invariant detector.
\item
  Inference of @NonNull: If your code uses the Nullable default (this is
  unusual), use the ``Non-null checker and inferencer'' of the JastAdd Extensible Compiler.
\item “Locking discipline inference and checking”
  \url{https://homes.cs.washington.edu/~mernst/pubs/locking-inference-checking-icse2016-abstract.html}
\item
   “Interval Type Inference: Improvements and Evaluations”
  \url{https://uwspace.uwaterloo.ca/bitstream/handle/10012/17788/Wang_Di.pdf}
\item
  “Precise inference of expressive units of measurement types” (OOPSLA
  2020) [XLD20].
\item
  “Inference of field initialization” [SE11]
\item
  SFlow is a context-sensitive type system for secure information flow. It
  contains two variants, SFlow/Integrity and
  SFlow/Confidentiality. SFlowInfer is its worst-case-cubic inference
  analysis.
\item
  Cascade [VPEJ15] is an Eclipse plugin that implements interactive type
  qualifier inference. Cascade is interactive rather than fully-automated:
  it makes it easier for a developer to insert annotations. Cascade starts
  with an unannotated program and runs a type-checker. For each warning it
  suggests multiple fixes, the developer chooses a fix, and Cascade applies
  it. Cascade works with any checker built on the Checker Framework. You
  can find installation instructions and a video tutorial at
  https://github.com/reprogrammer/cascade. Cascade was last updated in
  November 2014, so it might or might not work for you.
\end{itemize}
  


\section{Limitations and Future Work}
\label{sec:limits}

\todo{This section should discuss the threats to validity of our experiments.}

\todo{This section should discuss the limitations of our approach, especially
  that we cannot infer polymorphic types at all.}

\todo{This section should discuss future work, especially any ideas we have about
  inferring polymorphic types.}


\section{Related Work}
\label{sec:relatedwork}

We expect that other approaches could do as well or better than ours.  But
the difficulty (labor) has precluded their creation.  (Including by us.)


\subsection{Type Inference For Pluggable Type Systems}
\label{sec:rw:type-inference-pluggable}

Talk about specific type inference approaches for pluggable type systems, including those we
compared against in \cref{sec:comparison}. 
The best citation for Werner's work on CFI is \cite{XiangLD2020}.

Probably also CASCADE.

\subsection{Type Inference Approaches}
\label{sec:rw:type-inference}

Talk about general type inference approaches, classic and modern.

\subsection{Pluggable Types}
\label{sec:pluggable}

Talk about other approaches to making type systems easier to use?

\todo{Any other general category of things I'm missing here?}

\todo{Mike should write this up.}
Houdini also repeatedly runs a verification tool, until fixed point \cite{FlanaganJL01,FlanaganL2001:Houdini}.


\section{Conclusion}

\todo{This section concludes.}

%% In future work, we plan to develop
%% improved inference techniques for lightweight ownership annotations.  Since
%% these annotations can be added anywhere without impacting soundness (they
%% are verified, not trusted), genetic
%% search
%% and machine-learning techniques could be used to introduce them, using the
%% warnings emitted by \tool as the fitness function.


\begin{acks}
\todo{Find the list of people who have contributed code to WPI.}
\end{acks}

%% The next two lines define the bibliography style to be used, and
%% the bibliography file.
\bibliographystyle{ACM-Reference-Format}
\balance
\bibliography{bib/bibstring-abbrev,bib/types,bib/dispatch,bib/ernst,bib/soft-eng,bib/invariants,bib/crossrefs,temp}

%%
%% If your work has an appendix, this is the place to put it.

\end{document}
\endinput
%%
%% End of file `sample-sigplan.tex'.

% LocalWords:  Kushigian Chandrakana Nandi LoC CCF
