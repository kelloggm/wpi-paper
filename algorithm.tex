\section{Type Inference Algorithm}
\label{sec:algorithm}

This section presents our type inference algorithm. This algorithm is
polymorphic with respect to the underlying pluggable typechecker: that is,
it applies equally well to any pluggable typechecker that performs flow-sensitive
local inference. The algorithm is described in two parts: first, the fixpoint
algorithm used to infer types for a particular program (\cref{sec:core-algorithm});
and second, the modifications we make to the pluggable typechecker to enable
that algorithm (\ie the \textsc{Instrument} function of \cref{alg:wpi-fixpoint},
in \cref{sec:instrument}).

\subsection{Fixpoint Algorithm}
\label{sec:core-algorithm}

% the core WPI fixpoint algorithm; that is, the outer WPI loop

\begin{algorithm}
  \DontPrintSemicolon
  \SetKwFunction{Infer}{infer}
  \SetKwProg{Fn}{def}{:}{}
  \SetKwInOut{Input}{input}
  \SetKwInOut{Output}{output}
  \Input{program $P$ and pluggable typechecker $T$}
  \Output{set of errors $E$ and set of type qualifiers $A^{\prime}$}
  \Fn{ \Infer{$P$, $T$} } {
%    \tcc{In each iteration, $A$ is the specification set before running $T$, and $A^{\prime}$ is the set after.}
    $A \gets \emptyset, A^{\prime} \gets \emptyset$ \;
%    \tcc{\textsc{Instrument} modifies $T$ to both check the program and collect a candidate specification set.}
    $T^{\prime} \gets \textsc{EnableInference}(T)$ \;
    $E, A^{\prime} \gets T^{\prime}(P, A)$ \;
    \While{$E \neq \emptyset \wedge A \neq A^{\prime}$}{
      \tcc{Note that each element of $A^{\prime}$ is either a new type qualifier or refines some element of $A$.}
      $A \gets A^{\prime}$ \;
      $E, A^{\prime} \gets T^{\prime}(P, A)$ \;
    }
    \Return $E, A^{\prime}$ \;
  }
  \caption{The core fixpoint algorithm for our type inference system.
    The helper function \textsc{EnableInference} is defined by the modifications
    to the framework described in \cref{sec:instrument}.
    \todo{Why is the caption so narrow, not taking up the whole column?}
}
  \label{alg:wpi-fixpoint}
\end{algorithm}


This section presents the core fixpoint algorithm, which appears
in \cref{alg:wpi-fixpoint}.

\todo{Describe it.}

\subsubsection{Soundness}
\label{sec:soundness}

\Cref{alg:wpi-fixpoint} is sound, assuming the underlying typechecker $T$
is sound. The key reason is that \cref{alg:wpi-fixpoint} always runs $T$
on each set of proposed annotations. If any of the annotations are incorrect,
$T$ will report an error---in the same manner as if a human had written an
incorrect annotation. In particular, this means that the instrumentation
(\cref{sec:instrument}) has no obligation to produce annotations that are sound,
so long as the underlying typechecking algorithm itself is not modified.
\todo{Maybe formalize this a bit more?}

\subsection{Modification to the Typechecker}
\label{sec:instrument}

The algorithm presented in \cref{sec:core-algorithm} works for
any pluggable typechecker that supports flow-sensitive inference:
that is, it does not require a type system implementer to write
any special rules to support type inference. Instead, this section
describes our approach to \emph{automatically} modify a given
pluggable typechecker to support inference, corresponding to the
\textsc{Instrument} helper function in \cref{alg:wpi-fixpoint}.

\todo{Describe how this works. Include the changes to the
  general type rules for qualified types so that candidate
  specifications are also an output. Basically, there should be
  something in this section for every public method in
  \<WholeProgramInference>.}
