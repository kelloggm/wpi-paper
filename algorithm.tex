\section{Type Inference Algorithm}
\label{sec:algorithm}

This section presents our type inference algorithm. This algorithm is
independent of the underlying pluggable typechecker: that is,
it applies equally well to any pluggable typechecker that performs flow-sensitive
local inference.

\todo{I don't like the term ``instrumented''.  It does describe what
  happened at a low level.  It does not explain what the instrumentation
  does or why it is valuable.  Maybe an ``inferring'' version of the type checker?}
\todo{The paper should explain that ``specifications'' is really ``type
  annotations'', to a first approximation.}

The crux\todo{word choice} of the implementation is to use an
``instrumented'' version of the type checker.  A typechecker
$T : P \rightarrow E$
takes a program and outputs a (possibly empty) set of
type errors.  The instrumented type-checker
$T' : \langle P, A \rangle \rightarrow \langle E, A' \rangle$
takes a program along with extra specifications, and it outputs errors and
inferences.  The errors are exactly those $T$ would output, if the
specifications in $A$ were written on $P$.

The algorithm is described in two parts: first,
\Cref{sec:core-algorithm} gives the fixpoint
algorithm used to infer types for a particular program.
\Cref{alg:wpi-fixpoint} gives the modifications that convert a pluggable
typechecker $T$ into an instrumented typechecker $T'$---that is,
the \textsc{Instrument} function.


\subsection{Fixpoint Algorithm}
\label{sec:core-algorithm}

% the core WPI fixpoint algorithm; that is, the outer WPI loop

\begin{algorithm}
  \DontPrintSemicolon
  \SetKwInOut{Input}{input}
  \SetKwInOut{Output}{output}
  \Input{program $P$ and pluggable typechecker $T$}
  \Output{Set of specifications (inferred types) $A^{\prime}$}
  \BlankLine
%  \tcc{In each iteration, $A$ is the specification set before running $T$, and $A^{\prime}$ is the set after.}
  $A \gets \emptyset, A^{\prime} \gets \emptyset$ \;
%  \tcc{\textsc{Instrument} modifies $T$ to both check the program and collect a candidate specification set.}
  $T^{\prime} \gets \textsc{Instrument}(T)$ \;
  $E, A^{\prime} \gets T^{\prime}(P, A)$ \;
  \While{$E \neq \emptyset \wedge A \neq A^{\prime}$}{
    $A \gets A^{\prime}$ \;
    $E, A^{\prime} \gets T^{\prime}(P, A)$ \;
  }
  % \Return $E, A^{\prime}$ \;
  \Return $E, A^{\prime}$ \;

  \caption{The core fixpoint algorithm for our type inference system.
    The helper function \textsc{Instrument} is defined and explained in \cref{sec:instrument}.
\todo{Make this into a function and give it a name, probably
  \textsc{Infer}.}
\todo{Why is the caption so narrow, not taking up the whole column?}
\todo{before ``$A \gets A^{\prime}$'', add a comment stating that $A'
  \sqsubseteq A$.  Put a little proof of that fact in the body of the paper.}
}
  \label{alg:wpi-fixpoint}
\end{algorithm}


This section presents the core fixpoint algorithm, which appears
in \cref{alg:wpi-fixpoint}. The key idea is to iteratively analyze
the target program ($P$) with an instrumented version of the
pluggable typechecker, recording intermediate results at each
step (the sets of specifications $A$ and $A^{\prime}$) until
either there are no remaining typechecking errors (\ie $E \neq \emptyset$)
or the
specifications reach fixpoint (\ie $A = A^{\prime}$).

This fixpoint loop is quite general: its success for our
purposes depends heavily on the \textsc{Instrument} procedure
that outputs locally-inferred specifications. \Cref{sec:instrument}
describes how we implement \textsc{Instrument} in a way that is
applicable to any pluggable typechecker.

\subsubsection{Soundness}
\label{sec:soundness}

\todo{This needs to define ``sound''.  I think that absolutely any
  inference strategy is sound, when paired with a type-checking run
  afterward.  I think it might be easier to say that.}
\Cref{alg:wpi-fixpoint} is sound, assuming the underlying typechecker $T$
is sound. The key intuition is that \cref{alg:wpi-fixpoint} always runs $T$
on each set of proposed specifications. If any specification is incorrect,
$T$ will report an error---in the same manner as if a human had written an
incorrect specification. In particular, this means that the instrumentation
(\cref{sec:instrument}) has no obligation to produce annotations that are sound,
so long as the underlying typechecking algorithm itself is not modified.
\todo{Maybe formalize this a bit more?}

\subsection{Modifications to the Typechecker}
\label{sec:instrument}

The algorithm presented in \cref{sec:core-algorithm} works for
any pluggable typechecker that supports flow-sensitive inference:
that is, it does not require a type system implementer to write
any special rules to support type inference. Instead, this section
describes our approach to \emph{automatically} modify a given
pluggable typechecker to support inference, corresponding to the
\textsc{Instrument} helper function in \cref{alg:wpi-fixpoint}.

The key idea behind our inference approach to instrumenting the typechecker
is to do modify the \emph{framework}.  Once the pluggable type-checking
framework is modified, inference is enabled for every typechecker built on it.
Our modifications
can be conceptualized at the type-qualifier-theory level: that is,
we modify the rules for typechecking \emph{any} pluggable type system
so that inference is supported,
regardless of the particular qualifiers it happens to support.

Our modifications rely on the fact that practical pluggable type systems
do \emph{local}, intra-procedural flow-sensitive type inference.
This means that programmers rarely need to write annotations within method bodies.
Analogously, Java's \<var> keyword permits programmers to omit Java basetypes within
method bodies.
Programmers are more willing to write types on method
signatures, where they form valuable documentation.  (Our goal is to lift
even this burden, for type annotations)

This assumption is
reasonable in practice: programmers are generally not willing to
write type qualifiers within method bodies, and so 
the pluggable type frameworks that exist in practice \todo{all?}
support this feature~\cite{PapiACPE2008}~\todo{cite any other practical pluggable
  type frameworks that exist, if there are any}. \todo{Also mention that
  Java itself supports this now for the base type system?} \todo{Also mention
  that this seems to be a general trend in language design, citing maybe Kotlin?}

\begin{figure*}
  \begin{mathpar}
    \inferrule* [right=INV]
                {
                  \std{\Gamma \vdash m(f_0,\ldots,f_n) : }~\qual{q_R}~\std{\tau_R}
                  \\
                  \std{\Gamma \vdash \forall i \in 0,\ldots,n,~e_i :~} \qual{q_{i_A}}~\std{\tau_{i_A}}
                  \\
                  \std{\Gamma \vdash \forall i \in 0,\ldots,n,~f_i :~} \qual{q_{i_F}}~\std{\tau_{i_F}}
                  \\
                  \std{\Gamma \vdash \forall i \in 0,\ldots,n,~} \qual{q_{i_A}}~\std{\tau_{i_A}~\sqsubseteq}~\qual{q_{i_F}}~\std{\tau_{i_F}}
                  \\
                  \infr{\infEnv \vdash \forall i \in 0,\ldots,n,~f_i~:~q_{i_I}~\tau_{i_F}}
                }
                {
                  \std{\Gamma \vdash m(e_0,\ldots,e_n) : }~\qual{q_R}~\std{\tau_R}
                  \\
                  \infr{\infEnv \vdash \forall i \in 0,\ldots,n,~f_i~:~\mathit{LUB_Q}(q_{i_A},~q_{i_I})~\tau_{i_F} }
                }
                
     \inferrule* [right=NEW]
                {
                  \std{\Gamma \vdash \<new T>(f_1,\ldots,f_n) : }~\qual{q_R}~\std{\tau_R}
                  \\
                  \std{\Gamma \vdash \forall i \in 1,\ldots,n,~e_i :~} \qual{q_{i_A}}~\std{\tau_{i_A}}
                  \\
                  \std{\Gamma \vdash \forall i \in 1,\ldots,n,~f_i :~} \qual{q_{i_F}}~\std{\tau_{i_F}}
                  \\
                  \std{\Gamma \vdash \forall i \in 1,\ldots,n,~} \qual{q_{i_A}}~\std{\tau_{i_A}~\sqsubseteq}~\qual{q_{i_F}}~\std{\tau_{i_F}}
                  \\
                  \infr{\infEnv \vdash \forall i \in 1,\ldots,n,~f_i~:~q_{i_I}~\tau_{i_F}}
                }
                {
                  \std{\Gamma \vdash \<new T>(e_1,\ldots,e_n) : }~\qual{q_R}~\std{\tau_R}
                  \\
                  \infr{\infEnv \vdash \forall i \in 1,\ldots,n,~f_i~:~\mathit{LUB_Q}(q_{i_A},~q_{i_I})~\tau_{i_F} }
                }

     \inferrule* [right=FORMAL-ASSIGN]
                {
%                  \std{f~is~a~formal~parameter} \\
                  \std{\Gamma \vdash e~:}~\qual{q_A}~\std{\tau_A} \\
                  \std{\Gamma \vdash f~:}~\qual{q_F}~\std{\tau_F} \\
                  \std{\Gamma \vdash} \qual{q_A}~\std{\tau_A~\sqsubseteq}~\qual{q_F}~\std{\tau_F} \\
                  \infr{\infEnv \vdash f~:~q_I~\tau_F}
                }
                {
                  \std{\Gamma \vdash f~:=~e} \\
                  \infr{\infEnv \vdash f~:~\mathit{LUB_Q}(q_A, q_I)~\tau_F}
                }

     \inferrule* [right=FIELD-ASSIGN]
                {
%                  \std{f~is~a~formal~parameter} \\
                  \std{\Gamma \vdash e~:}~\qual{q_A}~\std{\tau_A} \\
                  \std{\Gamma \vdash x.f~:}~\qual{q_F}~\std{\tau_F} \\
                  \std{\Gamma \vdash} \qual{q_A}~\std{\tau_A~\sqsubseteq}~\qual{q_F}~\std{\tau_F} \\
                  \infr{\infEnv \vdash x.f~:~q_I~\tau_F}
                }
                {
                  \std{\Gamma \vdash x.f~:=~e} \\
                  \infr{\infEnv \vdash x.f~:~\mathit{LUB_Q}(q_A, q_I)~\tau_F}
                }

     \inferrule* [right=RETURN]
                {
%                  \std{f~is~a~formal~parameter} \\
                  \std{\Gamma \vdash e~:}~\qual{q_A}~\std{\tau_A} \\
                  \std{\Gamma \vdash m(f_0,\ldots,f_n)~:}~\qual{q_F}~\std{\tau_F} \\
                  \std{\Gamma \vdash} \qual{q_A}~\std{\tau_A~\sqsubseteq}~\qual{q_F}~\std{\tau_F} \\
                  \infr{\infEnv \vdash m(f_0,\ldots,f_n)~:~q_I~\tau_F}
                }
                {
                  \std{\Gamma \vdash \<return>~e} \\
                  \infr{\infEnv \vdash m(f_0,\ldots,f_n)~:~\mathit{LUB_Q}(q_A, q_I)~\tau_F}
                }

    \inferrule* [right=OVERRIDE]
                {
                  % return types
                  \std{\Gamma \vdash m_1(f_{0_1},\ldots,f_{n_1}) : }~\qual{q_{R_1}}~\std{\tau_{R_1}}
                  \\
                  \std{\Gamma \vdash m_2(f_{0_2},\ldots,f_{n_2}) : }~\qual{q_{R_2}}~\std{\tau_{R_2}}
                  \\
                  \std{\Gamma \vdash} \qual{q_{R_1}}~\std{\tau_{R_1}~\sqsubseteq}~\qual{q_{R_2}}~\std{\tau_{R_2}}
                  \\
                  \std{\Gamma \vdash \forall i \in 0,\ldots,n_1,~f_{i_1} :~} \qual{q_{i_1}}~\std{\tau_{i_1}}
                  \\
                  \std{\Gamma \vdash \forall i \in 0,\ldots,n_2,~f_{i_2} :~} \qual{q_{i_2}}~\std{\tau_{i_2}}
                  \\
                  \std{\Gamma \vdash \forall i \in 0,\ldots,n_1,~} \qual{q_{i_1}}~\std{\tau_{i_1}~\sqsubseteq}~\qual{q_{i_2}}~\std{\tau_{i_2}}
                  \\
                  \std{\vdash n_1~=~n_2}
                  \\
                  \infr{\infEnv \vdash m_1(f_{0_1},\ldots,f_{n_1})~:~q_{R_1-I}~\tau_{R_1}}
                  \\
                  \infr{\infEnv \vdash m_2(f_{0_2},\ldots,f_{n_2})~:~q_{R_2-I}~\tau_{R_2}}
                  \\
                  \infr{\infEnv \vdash \forall i \in 0,\ldots,n_1,~f_{i_1}~:~q_{i_1-I}~\tau_{i_1}}
                  \\
                  \infr{\infEnv \vdash \forall i \in 0,\ldots,n_2,~f_{i_2}~:~q_{i_2-I}~\tau_{i_2}}
                }
                {
                  \std{\Gamma \vdash m_1(f_{0_1},\ldots,f_{n_1})~\mathit{is~a~valid~override~of}~m_2(f_{0_2},\ldots,f_{n_2})}
                  \\
                  \infr{\infEnv \vdash m_2(f_{0_2},\ldots,f_{n_2})~:~\mathit{LUB_Q}(q_{R_1-I}, q_{R_2-I})~\tau_{R_2}}
                  \\
                  \infr{\infEnv \vdash \forall i \in 0,\ldots,n_2,~f_{i_2}~:~\mathit{LUB_Q}(q_{i_1-I},~q_{i_2-I})~\tau_{i_2} }
                }
                
                
  \end{mathpar}
  \caption{Modified type rules used by our pluggable type framework. \std{Gray} indicates
    standard type rules for a Java-like language. \qual{Black} indicates additions to support
    pluggable typechecking. \infr{Red} indicates additions to support inference, \ie our
    contribution in this paper.
    Throughout, ``R'' subscripts refer to return types; ``F'' to formal parameters; ``A'' to
    actual arguments; and ``I'' to inference results.
    Type rules that do not require modification to support inference
    are elided for space.}
  \label{fig:type-rules}
\end{figure*}

The modified type rules appear in \cref{fig:type-rules}. \std{Gray} indicates
standard type rules for a Java-like language. \qual{Black} indicates additions to support
pluggable typechecking. \infr{Red} indicates additions to support inference, \ie our
contribution in this paper.
%
Type rules that do not need to be modified to support inference are elided for space.
%
To read these type rules, we first need to define some terms.

$\infEnv$ is the \emph{inference environment}, similar to the standard (qualified)
typing environment $\Gamma$. $\Gamma$ maps expressions and declarations to qualified types.
$\infEnv$ maps declarations to possibly-qualified types.
$\infEnv$ only maps declarations because our inference procedure does not need to infer
types for expressions: in fact, we assume that $\Gamma$ already does so (via flow-sensitive
refinement). Rather, $\infEnv$'s purpose is to map declarations to the results of inference.
Unlike $\Gamma$, the values in $\infEnv$ are \emph{possibly-qualified}, meaning that they can either
be qualified types or unqualified types. Initially, $\infEnv$ contains qualified types only
for declarations that were qualified before the current round of inference (which may come from
the programmer or from a previous inference round in the \cref{alg:wpi-fixpoint} fixpoint loop).
Once the current round of inference terminates, $\infEnv$ is used to produce the results of the
inference round by returning the set of all qualified types: any type that remains unqualified
throughout inference is not annotated, because no information about it was learned.

A pluggable type \emph{checker} permits programmers to leave basetypes unqualified.
On APIs (class/method/field declarations), the type checker uses defaulting
rules to assign a qualifier to each unqualified basetype.
Within a code block, the type checker performs local flow-sensitive type
inference.


We define the function $\mathit{LUB_Q}(q_1, q_2)$ to account for possibly-qualified types.
$q_1$ and $q_2$ are each either a type qualifier or ``not present''.
If both arguments are qualifiers, then the result of $\mathit{LUB_Q}$
is just their least upper bound. If only one qualifier is present, then $\mathit{LUB_Q}$'s result
is that qualifier; if both qualifiers are not present, $\mathit{LUB_Q}$'s result is ``not present'',
resulting in an unqualified result.

\todo{Then, this section should describe the interesting parts of the
  modified type rules in detail? At a minimum, we should probably walk the reader through
  one of the type rules.}


\todo{Here is a fact that I think we should explain to readers.
  Within a single type-checking run, the type estimates only go up.
  Between runs (that is, in \cref{alg:wpi-fixpoint}), the type estimates
  only go down.}
