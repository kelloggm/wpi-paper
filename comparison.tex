\section{Comparison to Other Tools}
\label{sec:comparison}

In this section, we compare our generic approach to
checker-specific inference tools. We probably compare
on our nullness benchmarks against the nullness-specific
tools. We might compare against CFI.

We hope that the results show that WPI is about as good as
the other approaches. The relative advantage of WPI is that
it doesn't require any custom code, unlike the others.

A table presents the results.


Here are potential tools:
\begin{itemize}
\item
  Inference of \<@Nullable>: the AnnotateNullable tool of the Daikon invariant detector.
\item
  Inference of @NonNull: If your code uses the Nullable default (this is
  unusual), use the ``Non-null checker and inferencer'' of the JastAdd Extensible Compiler.
\item “Locking discipline inference and checking”
  \url{https://homes.cs.washington.edu/~mernst/pubs/locking-inference-checking-icse2016-abstract.html}
\item
   “Interval Type Inference: Improvements and Evaluations”
  \url{https://uwspace.uwaterloo.ca/bitstream/handle/10012/17788/Wang_Di.pdf}
\item
  “Precise inference of expressive units of measurement types” (OOPSLA
  2020) [XLD20].
\item
  “Inference of field initialization” [SE11]
\item
  SFlow is a context-sensitive type system for secure information flow. It
  contains two variants, SFlow/Integrity and
  SFlow/Confidentiality. SFlowInfer is its worst-case-cubic inference
  analysis.
\item
  Cascade [VPEJ15] is an Eclipse plugin that implements interactive type
  qualifier inference. Cascade is interactive rather than fully-automated:
  it makes it easier for a developer to insert annotations. Cascade starts
  with an unannotated program and runs a type-checker. For each warning it
  suggests multiple fixes, the developer chooses a fix, and Cascade applies
  it. Cascade works with any checker built on the Checker Framework. You
  can find installation instructions and a video tutorial at
  https://github.com/reprogrammer/cascade. Cascade was last updated in
  November 2014, so it might or might not work for you.
\end{itemize}
  
