\section{Evaluation}
\label{sec:evaluation}

We aim to answer two research questions:
\begin{itemize}
\item \textbf{RQ1:} is our approach \emph{effective}: can infer most of the
  annotations that humans have written in prior case studies of pluggable typecheckers?
\item \textbf{RQ2:} is our approach \emph{general}: without any extra per-checker work,
  can we apply it to many extant pluggable typecheckers?.
\end{itemize}

%% This is the primary table.

\begin{table*}
  \caption{
    WPI performance on human-annotated benchmarks. ``Annotated for'' is
    the names of the typecheckers for which human annotations were written
    (see \cref{tab:checkers} to understand the codes).
    ``NCNB LoC'' is the number of lines of non-comment, non-blank code.
    ``Annos.'' is the number of human-written annotations, across all typecheckers.
    ``WPI \% Inferred'' is the percentage of the human-written annotations that WPI
    inferred.\todo{What if WPI inferred something weaker or stronger?  How
      does that count?}\todo{Does it make sense to measure precision as
      well as recall?}\todo{Does it make sense to record the number of
      type-checking errors?  I don't think so:  the number of wrong
      inferred annotations is a better measure of remaining human effort.}
  }
  \label{tab:case-studies}
  \posttablecaption

  \begin{tabular}{@{}ll|rrr@{}}
    \textbf{\smaller{Benchmark}} & Annoted for & NCNB LoC & Annos. & WPI \% Inferred \\
    \textbf{\smaller{JFreeChart}} & In & 95,000 & 2,500 & X\% \\
    \textbf{\smaller{PlumeUtil}} & All & 10,000 & ? & X\% \\
    \ldots & \ldots & \ldots & \ldots & \ldots \\
  \end{tabular}
\end{table*}


We answer both of these research questions with one set of
experiments.
%
Our methodology (at a high-level) is to collect a set of
projects that have been annotated by humans so that some pluggable
typechecker can typecheck them,
remove the human-written type annotations,
and then apply our approach.
%
To answer \textbf{RQ1}, we compare the output of our approach to the
ground-truth human-written type annotations (\cref{sec:results}}.
%
\textbf{RQ2} is answered incidentally: \todo{no} modifications to
the overall approach were required to handle any of the \todo{number} typecheckers
we considered; we further discuss our approach's performance for different typecheckers
in \cref{sec:by-checker}.

\subsection{Methodology}
\label{sec:methodology}

\todo{Write this.}

\subsection{Results: Effectiveness}
\label{sec:results}

Our main results appear in \cref{tab:case-studies}. \todo{Discuss them.}

\subsection{Results: Generality}
\label{sec:by-checker}

%% This is the table per-checker.

\begin{table*}
  \caption{
    The results of \cref{tab:case-studies} indexed by typechecker
    rather than by benchmark. ``Code'' is the two-character\todo{Two is too
      short and cryptic.  Please expand.  Even 3 would be better!} code
    used in the ``Annotated for'' column of \cref{tab:case-studies}.
    A code in \cref{tab:case-studies} of ``All'' means that every
    checker in this table was run on that project.\todo{Do we need the
      ``All'' code?}
  }
  \label{tab:checkers}
  \posttablecaption

  \begin{tabular}{@{}ll|rrrr@{}}
    \textbf{\smaller{Checker}} & Code & \# projects & NCNB LoC & Annos. & WPI \% Inferred \\
    \textbf{\smaller{Index}} & In & 2 & 120,000 & 3,200 & X\% \\
    \textbf{\smaller{Nullness}} & Nu & 5 & 100,000 & ? & X\% \\
    \ldots & \ldots & \ldots & \ldots & \ldots \\
  \end{tabular}
\end{table*}

% LocalWords:  typechecker NCNB LoC Annos WPI


We also broke out the results by checker, in \cref{tab:checkers}. \todo{Discuss
  the results, especially if there are significant differences between checkers..}

\todo{Also emphasize the absolute number of checkers that WPI applies to.}
